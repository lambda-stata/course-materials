%\input{tcilatex}
\documentclass{article}
\usepackage[flushleft]{threeparttable}
%\usepackage[nolists,heads]{endfloat}
\usepackage{pdflscape}
\usepackage{amssymb}
\usepackage{dcolumn}
\usepackage{multirow}
%\usepackage{slashbox}
\usepackage{longtable}
\usepackage{booktabs}
\usepackage{setspace}
\usepackage[skip = 0pt]{caption}
\captionsetup{justification=centering}
\usepackage{subcaption}
\usepackage{footnote}
\usepackage{fullpage}
\usepackage{mathrsfs,amsfonts}
\usepackage{amsmath}
\usepackage{graphicx}
\usepackage{float}
\usepackage{changepage}
\usepackage{tabularx}
\usepackage{siunitx}
\usepackage[table]{xcolor}
\definecolor{airforceblue}{rgb}{0.36, 0.54, 0.66}
\definecolor{olivine}{rgb}{0.6, 0.73, 0.45}
\usepackage{array}
\usepackage[plainpages=false,pdfpagelabels]{hyperref}
\usepackage[english]{babel}
\usepackage[utf8]{inputenc}
\usepackage{nameref}
%\usepackage[all]{hypcap}
%\usepackage{ctable}
\usepackage [autostyle, english = american]{csquotes}
\usepackage{apacite}
\usepackage{rotating}
\usepackage[T1]{fontenc}
\usepackage{lscape}
\usepackage{adjustbox}
\usepackage{afterpage}
\usepackage{booktabs}
\usepackage{lipsum}
\usepackage{geometry}

%font
\usepackage[sc]{mathpazo}

%DAG
\usepackage{tikz}
\usetikzlibrary{positioning}
%\tikzset{mynode/.style={draw,text width=1in,align=center}}
\tikzset{mynode/.style={draw,align=center}}

\makeatletter\let\expandableinput\@@input\makeatother
\MakeOuterQuote{"}
\setcounter{MaxMatrixCols}{10}
\newtheorem{acknowledgement}{Acknowledgement}
\newtheorem{algorithm}{Algorithm}
\newtheorem{axiom}{Axiom}
\newtheorem{case}{Case}
\newtheorem{claim}{Claim}
\newtheorem{conclusion}{Conclusion}
\newtheorem{condition}{Condition}
\newtheorem{conjecture}{Conjecture}
\newtheorem{corollary}{Corollary}
\newtheorem{criterion}{Criterion}
\newtheorem{definition}{Definition}
\newtheorem{example}{Example}
\newtheorem{exercise}{Exercise}
\newtheorem{lemma}{Lemma}
\newtheorem{notation}{Notation}
\newtheorem{problem}{Problem}
\newtheorem{proposition}{Proposition}
\newtheorem{remark}{Remark}
\newtheorem{solution}{Solution}
\newtheorem{assumption}{Assumption}

% new environment for landscape tables
\newenvironment{ltable}{\begin{landscape}\begin{table}}{\end{table}\end{landscape}}
\newenvironment{ltablelong}{\begin{landscape}\begin{longtable}}{\end{longtable}\end{landscape}}

\newcolumntype{H}{>{\setbox0=\hbox\bgroup}c<{\egroup}@{}}
\newcolumntype{P}[1]{>{\centering\arraybackslash}p{#1}}

\newcommand\independent{\protect\mathpalette{\protect\independenT}{\perp}}
\def\independenT#1#2{\mathrel{\rlap{$#1#2$}\mkern2mu{#1#2}}}

\makeatletter
\newcommand\primitiveinput[1]
{\@@input #1 }
\makeatother

% tables numbers setup
%\numberwithin{table}{section}

% colors
\usepackage{colortbl}
\usepackage{url}
\urlstyle{rm}
\definecolor{darkblue}{rgb}{0,0,.4}
\hypersetup{colorlinks=true,
			breaklinks=true,
			citecolor=darkblue,
			linkcolor=darkblue,
			menucolor=darkblue,
			urlcolor=darkblue}

% Tables new command
\newcommand{\returnupdates}{%
Return to \nameref{sec:updates}.%
}

%-----------------------------------------------------------------
% BEGIN
%-----------------------------------------------------------------
\begin{document}
%-----------------------------------------------------------------
\title{Sesión 7: Exportación de Tablas \\ Card and Krueger (1994)}
\author{Rony Rodriguez-Ramirez}
\maketitle
%-----------------------------------------------------------------
% Sección: Tablas
%-----------------------------------------------------------------
\newpage
\section{Tablas}
%-----------------------------------------------------------------
% Tabla 2
%-----------------------------------------------------------------
\setcounter{table}{1}

\begin{table}[H]
	\centering
	\label{tab:Table}
		\begin{threeparttable}
			\caption{Mean of Key Variables}
			\begin{tabular}{@{}l*{3}{c}@{}}
                \toprule
							& \multicolumn{2}{c}{Stores in:} & 						\\ \cmidrule(lr){2-3}
				Variable 	& NJ & PA & $tâ$										\\
				\midrule
				1. \textit{Distribution of Store Types (percentages):} 	& & & 	\\
				\primitiveinput{tablas/tab2_1.tex}
																		& & & 	\\
				2. \textit{Means in Wave 2:}							& & & 	\\
																		& & & 	\\
				\primitiveinput{tablas/tab2_2.tex}
																		& & &	\\
				3. \textit{Means in Wave 2:}							& & & 	\\
																		& & & 	\\
				\primitiveinput{tablas/tab2_3.tex}
				\bottomrule
			\end{tabular}
			\begin{tablenotes}
				\setlength\labelsep{0pt}
				\footnotesize
				\item \textit{Notes}: See text for definitions. Standard errors are given in parentheses.
				\item ~~$^a$~Test of equality of means in New Jersey and Pennsylvania.
			\end{tablenotes}
		\end{threeparttable}
\end{table}
%-----------------------------------------------------------------
% Tabla 4
%-----------------------------------------------------------------
\setcounter{table}{3}

\begin{table}[H]
	\centering
	\label{tab:Table2}
		\begin{threeparttable}
			\caption{Reduced-form modles for change in employment}
			\begin{tabular}{@{}l*{5}{c}@{}}
				\toprule
				& \multicolumn{5}{c}{Model}								\\ \cmidrule(l){2-6}
				Indendent variable & (i) & (ii) & (iii) & (iv) & (v) 	\\ \midrule
				\primitiveinput{tablas/tab4.tex}
				\bottomrule
			\end{tabular}
			\begin{tablenotes}
				\setlength\labelsep{0pt}
				\footnotesize
				\item \textit{Notes}: Standard errors are given in parentheses.
				The sample consists of 357 stores with available data on employment and starting wages in waves 1 and 2.
				The dependent variable in all models is change in FTE employment. The mean and standard deviation of
				the dependent variable are -0.237 and 8.825, respectively. All models include a n unrestricted constant (not reported).
				\item ~~~$^{a}$Proportional increase in starting wage necessary to raise starting wage to new minimum rate. For stores in Pennsylvania the wage gap is 0
				\item ~~~$^{b}$Three dummy variables for chain type and whether or not the store is company owned are included.
				\item ~~~$^{c}$Dummy variables for two regions of New Jersey and two regions of eastern Pennsylvania are included.
				\item ~~~$^{d}$Probability value of joint F test for exclusion of all control variables.
			\end{tablenotes}
		\end{threeparttable}
\end{table}
%-----------------------------------------------------------------
% END ------------------------------------------------------------
%-----------------------------------------------------------------
\end{document}
%-----------------------------------------------------------------
