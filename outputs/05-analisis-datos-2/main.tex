\documentclass{article}
%-----------------------------------------------------------------
% PREAMBLE -------------------------------------------------------
%-----------------------------------------------------------------
\usepackage[flushleft]{threeparttable}						
%\usepackage[nolists,heads]{endfloat}
\usepackage{pdflscape}
\usepackage{amssymb}
\usepackage{dcolumn}
\usepackage{multirow}
\usepackage{longtable}
\usepackage{booktabs}
\usepackage{setspace}
\usepackage[skip = 0pt]{caption}
\captionsetup{justification=centering}
\usepackage{subcaption}
\usepackage{footnote}
\usepackage{fullpage}
\usepackage{mathrsfs,amsfonts}
\usepackage{amsmath}
\usepackage{graphicx}
\usepackage{float}
\usepackage{changepage}
\usepackage{tabularx}
\usepackage{siunitx}
\usepackage[table]{xcolor}
\definecolor{airforceblue}{rgb}{0.36, 0.54, 0.66}
\definecolor{olivine}{rgb}{0.6, 0.73, 0.45}
\usepackage{array}
\usepackage[plainpages=false,pdfpagelabels]{hyperref}
\usepackage[english]{babel}
\usepackage[utf8]{inputenc}
\usepackage{nameref}
\usepackage [autostyle, english = american]{csquotes}
\usepackage{apacite}
\usepackage{rotating}
\usepackage[T1]{fontenc}
\usepackage{lscape}
\usepackage{adjustbox}
\usepackage{afterpage}
\usepackage{booktabs}
\usepackage{lipsum}
%-----------------------------------------------------------------
%font
%-----------------------------------------------------------------
\usepackage[sc]{mathpazo}
%-----------------------------------------------------------------
%DAG
%-----------------------------------------------------------------
\usepackage{tikz}
\usetikzlibrary{positioning}
%\tikzset{mynode/.style={draw,text width=1in,align=center}}
\tikzset{mynode/.style={draw,align=center}}
%-----------------------------------------------------------------
\makeatletter\let\expandableinput\@@input\makeatother
\MakeOuterQuote{"}
\setcounter{MaxMatrixCols}{10}
\newtheorem{acknowledgement}{Acknowledgement}
\newtheorem{algorithm}{Algorithm}
\newtheorem{axiom}{Axiom}
\newtheorem{case}{Case}
\newtheorem{claim}{Claim}
\newtheorem{conclusion}{Conclusion}
\newtheorem{condition}{Condition}
\newtheorem{conjecture}{Conjecture}
\newtheorem{corollary}{Corollary}
\newtheorem{criterion}{Criterion}
\newtheorem{definition}{Definition}
\newtheorem{example}{Example}
\newtheorem{exercise}{Exercise}
\newtheorem{lemma}{Lemma}
\newtheorem{notation}{Notation}
\newtheorem{problem}{Problem}
\newtheorem{proposition}{Proposition}
\newtheorem{remark}{Remark}
\newtheorem{solution}{Solution}
\newtheorem{assumption}{Assumption}
%-----------------------------------------------------------------
% new environment for landscape tables    
\newenvironment{ltable}{\begin{landscape}\begin{table}}{\end{table}\end{landscape}}
\newenvironment{ltablelong}{\begin{landscape}\begin{longtable}}{\end{longtable}\end{landscape}}
%-----------------------------------------------------------------
\newcolumntype{H}{>{\setbox0=\hbox\bgroup}c<{\egroup}@{}}
\newcolumntype{P}[1]{>{\centering\arraybackslash}p{#1}}
%-----------------------------------------------------------------
\newcommand\independent{\protect\mathpalette{\protect\independenT}{\perp}}
\def\independenT#1#2{\mathrel{\rlap{$#1#2$}\mkern2mu{#1#2}}}

%-----------------------------------------------------------------
\makeatletter
\newcommand\primitiveinput[1]
{\@@input #1 }
\makeatother
%-----------------------------------------------------------------
\def\sym#1{\ifmmode^{#1}\else\(^{#1}\)\fi}
%-----------------------------------------------------------------
% tables numbers setup
%\numberwithin{table}{section}
%-----------------------------------------------------------------
% colors
%-----------------------------------------------------------------
\usepackage{colortbl}
\usepackage{url}
\urlstyle{rm}
\definecolor{darkblue}{rgb}{0,0,.4}
\hypersetup{colorlinks=true, 
			breaklinks=true, 
			citecolor=darkblue, 
			linkcolor=darkblue, 
			menucolor=darkblue, 
			urlcolor=darkblue}
%-----------------------------------------------------------------
\begin{document}
%-----------------------------------------------------------------
\title{Stata Tables}
\author{Rony Rodriguez-Ramirez \\ LAMBDA}
\maketitle
\listoftables
%-----------------------------------------------------------------
\newpage
%-----------------------------------------------------------------
% FIRST TABLE
%-----------------------------------------------------------------
Esta sería mi tabla con mis formatos de esttab.

\begin{table}[H]
	\centering
	\label{tab:Table}
	\begin{adjustbox}{max width=\linewidth}
		\begin{threeparttable}
			\caption{Basic exported table}
			\begin{tabular}{@{}l*{4}{c}@{}}
        \toprule
        \toprule 
        & Mean & SD & Min & Max \\
				& (1) & (2) & (3) & (4) \\
				\primitiveinput{tablas/Table-esttab-example.tex}
				\bottomrule
			\end{tabular}
			\begin{tablenotes}
				\setlength\labelsep{0pt}
				\footnotesize
				\item \textit{Notes}: Algunas notas acá.
			\end{tablenotes}
		\end{threeparttable}
	\end{adjustbox}
\end{table}
%-----------------------------------------------------------------

Esta sería mi segunda tabla pero de una manera más sencilla.

\begin{table}[H]
  \centering
	\caption{Basic \texttt{esttab} table}
	\begin{adjustbox}{max width = \textwidth}
		{
\def\sym#1{\ifmmode^{#1}\else\(^{#1}\)\fi}
\begin{tabular}{l*{1}{cccc}}
\hline\hline
                    &\multicolumn{4}{c}{}                               \\
                    &        mean&          sd&         min&         max\\
\hline
Cultivation         &        0.50&        0.50&        0.00&        1.00\\
Irrigation          &        0.24&        0.43&        0.00&        1.00\\
Semila: Plot P in Season S of Crop C&       10.47&        5.48&        1.00&       20.00\\
Cosecha: Plot P in Season S of Crop C&       25.62&       14.15&        1.01&       50.00\\
Consumo: Plot P in Season S of Crop C&       17.35&       11.53&        1.00&       48.99\\
Venta: Plot P in Season S of Crop C&        8.27&       11.58&        0.00&       48.53\\
Ingreso por plot season crop&      131.52&      122.26&        0.04&      703.33\\
\hline
Observations        &       24408&            &            &            \\
\hline\hline
\end{tabular}
}

	\end{adjustbox}
\end{table}

Tabla de correlación:

\begin{table}[H]
  \centering
  \caption{Ejemplo de exportación de correlaciones}
  \begin{adjustbox}{max width = \textwidth}
    {
\def\sym#1{\ifmmode^{#1}\else\(^{#1}\)\fi}
\begin{tabular}{l*{4}{c}}
\hline\hline
          &\multicolumn{4}{c}{(1)}                                                    \\
          &\multicolumn{4}{c}{}                                                       \\
          &  seed\_kg         &  harv\_kg         &consum\_kg         &  sell\_kg         \\
\hline
seed\_kg   &        1         &                  &                  &                  \\
harv\_kg   &  0.00597         &        1         &                  &                  \\
consum\_kg &0.0000583         &    0.610\sym{***}&        1         &                  \\
sell\_kg   &  0.00724         &    0.614\sym{***}&   -0.251\sym{***}&        1         \\
\hline\hline
\multicolumn{5}{l}{\footnotesize \sym{*} \(p<0.05\), \sym{**} \(p<0.01\), \sym{***} \(p<0.001\)}\\
\end{tabular}
}

  \end{adjustbox}
\end{table}

Otra tabla de correlación:

\begin{table}[H]
	\centering
	\label{tab:Table}
	\begin{adjustbox}{max width=\linewidth}
		\begin{threeparttable}
			\caption{Ejemplo de exportación de correlaciones con estilo}
			\begin{tabular}{@{}l*{4}{c}@{}}
        \toprule
        \toprule 
				\primitiveinput{tablas/corr_example2.tex}
				\bottomrule
			\end{tabular}
			\begin{tablenotes}
				\setlength\labelsep{0pt}
				\footnotesize
				\item \textit{Notes}: Algunas notas acá.
			\end{tablenotes}
		\end{threeparttable}
	\end{adjustbox}
\end{table}


Tabla de regresiones:

\begin{table}[H]
	\centering
	\label{tab:Table2}
	\begin{adjustbox}{max width=\linewidth}
		\begin{threeparttable}
			\caption{Basic exported table with regressions}
			\begin{tabular}{@{}l*{4}{c}@{}}
        \toprule
        \toprule 
				& (1) & (2) & (3) & (4) \\
				\primitiveinput{tablas/esttab_reg.tex}
				\bottomrule
			\end{tabular}
			\begin{tablenotes}
				\setlength\labelsep{0pt}
				\footnotesize
				\item \textit{Notes}: Algunas notas acá.
			\end{tablenotes}
		\end{threeparttable}
	\end{adjustbox}
\end{table}

%-----------------------------------------------------------------
\end{document}
%-----------------------------------------------------------------