\documentclass[11pt, aspectratio=169, compress]{beamer}
\usetheme[progressbar=frame title, numbering=fraction]{metropolis}      % Use metropolis theme 
\setbeamertemplate{section in toc}[sections numbered]
\setbeamertemplate{subsection in toc}[subsections numbered]
\useoutertheme[subsection=false]{miniframes}
\setbeamercolor{section in head/foot}{fg=white, bg=mDarkTeal}
\setbeamercolor{background canvas}{bg=white}
\setbeamerfont{section in head/foot}{series=\bfseries}

\usefonttheme[onlymath]{serif}
\usepackage{amsmath}
\usepackage{remreset}
\usepackage{ragged2e}
\usepackage{booktabs}
\usepackage{makecell}
\usepackage{float}
\usepackage{subfig}
\usepackage{tikz}
\usetikzlibrary{positioning,calc}
\usepackage[flushleft]{threeparttable}	% 3 part table 
\usepackage[justification=centering]{caption}
\captionsetup{skip=0pt}
\graphicspath{{./fig/}}

\makeatletter
\let\beamer@writeslidentry@miniframeson=\beamer@writeslidentry
\def\beamer@writeslidentry@miniframesoff{%
	\expandafter\beamer@ifempty\expandafter{\beamer@framestartpage}{}% does not happen normally
	{%else
		% removed \addtocontents commands
		\clearpage\beamer@notesactions%
	}
}
\newcommand*{\miniframeson}{\let\beamer@writeslidentry=\beamer@writeslidentry@miniframeson}
\newcommand*{\miniframesoff}{\let\beamer@writeslidentry=\beamer@writeslidentry@miniframesoff}
\beamer@compresstrue
\makeatother

%==============================================================
% Title Page
%==============================================================
%Information to be included in the title page:
\title{Manejo y Limpieza de Datos}
\author{Rony Rodriguez-Ramírez} 
\institute{LAMBDA}
\titlegraphic{\hfill\includegraphics[height=1.5cm]{dime}}
\date{\today}
%==============================================================
\begin{document}
	
\begin{frame}[plain]
	\maketitle 
\end{frame}

%------------------------------------------------
\section{Manejo y Limpieza de Datos}
%-----------------------------------------------
\subsection{Manejo y Limpieza de Datos}
%-----------------------------------------------
\begin{frame}{Introducción}
	Reflexiones sobre la case anterior: 
	\begin{itemize}
		\item 
	\end{itemize}
\end{frame}
%-----------------------------------------------
\begin{frame}{¿Qué es un script maestro?}
	\begin{itemize}
		\item El script maestro es el mapa sobre todo el trabajo de datos en su carpeta de datos
		\item Es la tabla de contenido para las instrucciones que codifica
		\item Debería ser posible seguir todo el trabajo de datos en la carpeta de datos, desde datos sin procesar hasta resultados de análisis, leyendo el script maestro. 
	\end{itemize}
\end{frame}
%-----------------------------------------------
\begin{frame}{Script maestro: permite una colaboración fácil}
	\begin{itemize}
		\item Si compartimos un proyecto a través de DropBox o GitHub, todos los miembros del equipo tienen la misma estructura de carpetas.
		\item Un script maestro permite que varias personas establezcan su propio global en la carpeta del proyecto.
		\item De esta manera, cualquiera que comparta la carpeta del proyecto puede ejecutar fácilmente sus scripts.
	\end{itemize}
\end{frame}
%-----------------------------------------------


\end{document}		
