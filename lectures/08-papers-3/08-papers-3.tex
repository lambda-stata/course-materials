\documentclass[10pt, aspectratio=169, compress]{beamer}
\usetheme[progressbar=frame title, numbering=fraction]{metropolis}      % Use metropolis theme 
\setbeamertemplate{section in toc}[sections numbered]
\setbeamertemplate{subsection in toc}[subsections numbered]
\useoutertheme[subsection=false]{miniframes}
\setbeamercolor{section in head/foot}{fg=white, bg=mDarkTeal}
\setbeamercolor{background canvas}{bg=white}
\setbeamerfont{section in head/foot}{series=\bfseries}

\usefonttheme[onlymath]{serif}
\usepackage{amsmath}
\RequirePackage{silence}
\WarningFilter{remreset}{The remreset package}
\usepackage{ragged2e}
\usepackage{booktabs}
\usepackage{makecell}
\usepackage{float}
\usepackage{subfig}
\usepackage{tikz}
\usetikzlibrary{positioning,calc,trees}
\usepackage[flushleft]{threeparttable}	% 3 part table 
\usepackage[justification=centering]{caption}
\captionsetup{skip=0pt}
\graphicspath{{./fig/}}

% colors
\usepackage{listings}
\usepackage{colortbl}
\usepackage{url}
\urlstyle{rm}
\definecolor{darkblue}{rgb}{0,0,.4}

\makeatletter
\let\beamer@writeslidentry@miniframeson=\beamer@writeslidentry
\def\beamer@writeslidentry@miniframesoff{%
	\expandafter\beamer@ifempty\expandafter{\beamer@framestartpage}{}% does not happen normally
	{%else
		% removed \addtocontents commands
		\clearpage\beamer@notesactions%
	}
}
\newcommand*{\miniframeson}{\let\beamer@writeslidentry=\beamer@writeslidentry@miniframeson}
\newcommand*{\miniframesoff}{\let\beamer@writeslidentry=\beamer@writeslidentry@miniframesoff}
\beamer@compresstrue
\makeatother

%==============================================================
% Title Page
%==============================================================
%Information to be included in the title page:
\title{Exportando tablas en Stata III \\ Programando comandos | RDD}
\subtitle{Replicación de Papers}
\author{Rony Rodriguez-Ramírez} 
\institute{LAMBDA}
\titlegraphic{\hfill\includegraphics[height=1.5cm]{dime}}
\date{\today}
%==============================================================
\begin{document}
%------------------------------------------------	
\begin{frame}[plain]
	\maketitle 
\end{frame}
%------------------------------------------------
\section{Intro to Git}
%-----------------------------------------------
\subsection{Intro to Git}
%-----------------------------------------------
\begin{frame}
	\frametitle{Antes de la sesion:}

	\begin{itemize}
		\item Confirmemos que tienes una cuenta en GitHub: \hyperlink{https://github.com/join}{https://github.com/join}
		\item Instalar GitKraken: \hyperlink{https://www.gitkraken.com/}{https://www.gitkraken.com/}
	\end{itemize}
\end{frame}
%-----------------------------------------------
\begin{frame}
	\frametitle{Intro to Git}	
	\begin{figure}[H]
		\centering
		\includegraphics[width=0.35\textwidth]{finaldoc.pdf}
	\end{figure}
\end{frame}
%-----------------------------------------------
\begin{frame}
	\frametitle{¿Por qué Git?}

	\begin{itemize}
		\item Nos permite hacer rastreo de una documento. 
		\item ¿Quién lo hizo? ¿Qué modificaciones pasaron? ¿Por qué se hicieron estos cambios?
		\item Nos facilita colaboración entre personas.
		\item Podemos eliminar el problema de ``copia conflictiva'' en Dropbox. 
		\item Podemos configurar nuestro flujo de trabajo para replicaciones.
	\end{itemize}
\end{frame}
%-----------------------------------------------
\begin{frame}
	\frametitle{Intro to Git}

	Tres conceptos claves en Git:

	\begin{itemize}
		\item Clonar (Clone)
		\item Cometer (Commit)
		\item Rama (Branch)
	\end{itemize}
\end{frame}
%-----------------------------------------------
\begin{frame}
	\frametitle{¿Qué es clonar}

	Clonar es similar a descargar un repositorio en su computadora

	La diferencia entre clonar y descargar radica en que Git clona una repositorio y recuerda de donde se ha descargado este repositorio. Esto es necesario ya que Git conoce donde se debe compartir las ediciones que se hacen a una archivo específico de una repositorio.
\end{frame}
%-----------------------------------------------
\begin{frame}
	\frametitle{Clonar un repositorio}

	\begin{center}
		¿Cómo clonar un repositorio? \\ Live Preview
	\end{center}
	
\end{frame}
%-----------------------------------------------
\begin{frame}
	\frametitle{Explorando el clon}

	Dos puntos a tomar en cuenta en torno a los repositorios: 

	\begin{itemize}
		\item Commits
		\item Branches
	\end{itemize}

\end{frame}
%-----------------------------------------------
\begin{frame}
	\frametitle{¿Qué es una commit?}

	\begin{itemize}
		\item En lugar de tener una lista de cada versión guardada de un archivo, en Git usamos commits para indicar cuál es cada diferencia significativa entre dos versiones de nuestra carpeta de proyecto.
		\item Cada commit es una snapshot de todos los archivos de un proyecto. Git lista cada snapshot y lo compara con el snapshot anterior.
		\item Cada commit tiene una tiempo y rastreo de quién hizo el commit.  
	\end{itemize}

\end{frame}
%-----------------------------------------------
\begin{frame}
	\frametitle{¿Como hacer un commit?}

	\begin{center}
		Live Preview
	\end{center}

\end{frame}
%-----------------------------------------------
\begin{frame}
	\frametitle{Introduciendo branches (ramas)}

	\begin{columns}
		\begin{column}{0.5\textwidth}
			\begin{itemize}
				\item La rama es la \textbf{\textit{característica clave}} clave de Git.
				\item Las ramas nos permite crea una copia del código donde nosotros podemos experimentar. Si nos gusta el resultado podemos unir nuestro experimento con el projecto principal.
			\end{itemize}
		\end{column}
		\begin{column}{0.5\textwidth}
			\begin{center}
			 \includegraphics[width=1\textwidth]{git.png}
			 \end{center}
		\end{column}
	\end{columns}

\end{frame}
%-----------------------------------------------
\begin{frame}
	\frametitle{Más materiales}

	\begin{itemize}
		\item Lamentablemente por tiempo no podremos introducir más features de Git. 
	\end{itemize}
\end{frame}
%-----------------------------------------------
\section{Regresión discontinua}
\subsection{Regresión discontinua}
%-----------------------------------------------
\begin{frame}
	\frametitle{Intro: RDD}

	\begin{itemize}
		\item Presentado por primera vez para estudiar el impacto del reconocimiento al mérito por Thistlethwaite \& Campbell (1960).
		\item Sin embargo, solo comenzó a llamar la atención en economía desde finales de la década de 1990.
		\item Pero ¿qué es una discontinuidad?
		\begin{itemize}
			\item Una ruptura brusca en los valores de una función.
			\item Matemáticamente, estamos hablando de una ecuación por partes (piecewise equation):
			$$
			f(x) = \left\{
			\begin{array}{ll}
			\frac{1}{2} x, 	& \quad x < 5 \\
			2+\frac{1}{2}x,  & \quad x \geq 5
			\end{array}
			\right.
			$$
		\end{itemize}
	\end{itemize}
\end{frame}
%-----------------------------------------------
\begin{frame}
	\frametitle{Uso de RDD}

	\begin{itemize}
		\item No tenemos un RCT y nos preocupan las variables endógenas.
		\item RDD utiliza la asignación de discontinuidad exógena para estimar los efectos causales; es decir, las observaciones de un lado de la ruptura terminan siendo muy similares a las del otro lado.
		\item Esencialmente, terminamos con grupos equilibrados pero no tenemos una asignación de tratamiento aleatorizada.
	\end{itemize}

\end{frame}
%-----------------------------------------------
\begin{frame}
	\frametitle{Prinicipios de RDD}

	Randomización local: 

	\begin{itemize}
		\item El estado de tratamiento es una función determinista de una variabla $a$, por lo cual cuando conocemos $X$ conocemos el estado de tratamiento $D_x$. 
		\item El estado de tratamiento es una función discontinua de $a$ porque no importa que tan cerca X llegue al corte, $D_x$ continua sin cambios hasta que el corte se ha alcanzado.
		$$
		D_x = 
		\begin{cases}
		1 &\text{if}~X \geq \text{cutoff} \\  
		0 &\text{if}~X < \text{cutoff} 
		\end{cases}
		$$
	\end{itemize}

\end{frame}
%-----------------------------------------------
\begin{frame}
	\frametitle{Prinicipios de RDD}

	\begin{itemize}
		\item Llamamos a nuestra variable de asignación $X$, variable en ejecución (forcing or running variable).
		\item Un punto importante, es que esta variable no es ortogonal a:
		\begin{itemize}
			\item Las características observables de los individuos. 
			\item Las características no observables de los individuos.
		\end{itemize}
	\end{itemize}
	

\end{frame}
%-----------------------------------------------
\begin{frame}
	\frametitle{RD como una randomización local}

	\begin{figure}
		\centering
		\includegraphics[width=0.7\textwidth]{rdd.pdf}
	\end{figure}

\end{frame}
%-----------------------------------------------
\begin{frame}
	\frametitle{Dos condiciones claves en RDDs}

	\begin{enumerate}
		\item El tratamiento en la población debe depender en que si la variable observada excede el valor crítico denota como $c$.
		\item Los individuos no tiene un control preciso de la variable de asignación.
		\begin{itemize}
			\item "Sin control preciso" significa que entre los que puntúan cerca del umbral, es cuestión de "suerte" en cuanto a qué lado del umbral aterrizan.
			\item Las personas que marginalmente pasan el corte o quedan atras, son asumidos como identicos (bastante similares).
		\end{itemize}
	\end{enumerate}

\end{frame}
%-----------------------------------------------

%-----------------------------------------------
\section{Programación en Stata}
\subsection{Programación en Stata}
%-----------------------------------------------

\begin{frame}[t, fragile]{Creando nuestros comandos}
  Cuando escribe un comando que Stata no reconoce, Stata primero busca en su memoria un programa con ese nombre. Si Stata lo encuentra, Stata ejecuta el programa.
  
  Si Stata no encuentra el comando, debemos definirlo o instalarlo. Por ejemplo:

  \begin{lstlisting}
    . hello
    command hello is unrecognized
    r(199);
  \end{lstlisting}
  Así es como funciona la programación en Stata. Un programa se define por: 

  \begin{lstlisting}
    program define progname
      Stata commands
    end
  \end{lstlisting}
\end{frame}
%-----------------------------------------------
\begin{frame}[t, fragile]{Creando nuestros comandos}
  Si el comando que deseamos crear usa la sintaxis estándar de Stata. Es decir, argumentos como una lista de variables, posiblemente un peso, tal vez una cláusula "if" o "in", y tal vez un montón de opciones, podemos aprovechar el propio analizador de Stata, que almacena convenientemente todos estos elementos en macros locales listos para que los use.

  Entre program y end, debemos utilizar el comando \textit{syntax}. El cual tendrá podría tener la siguiente lógica:
  \begin{lstlisting}
    capture program drop test
    program define test
      syntax varname, options
      ...
    end
  \end{lstlisting}
\end{frame}
%-----------------------------------------------

\begin{frame}[t, fragile]{Creando nuestros comandos}
  Con varname estaremos especificando el uso de variables después de nuestro comando. Este es nuestro primer elemento.
  \begin{itemize}
    \item Puede especificar mínimos y máximos, por ejemplo, un programa que requiera exactamente dos variables diría varlist(min=2 max=2). 
    \item Cuando solo tiene una variable podemos escribir varname, que es la abreviatura de varlist(min=1 max=1).
  \end{itemize}
  Stata luego se asegurará de que su programa se llame con exactamente un nombre de una variable existente, que se almacenará en una macro local llamada \textit{varlist}.
\end{frame}
%-----------------------------------------------

\begin{frame}[t, fragile]{Creando nuestros comandos: [Opciones]}
  Los elementos de sintaxis opcionales están encerrados entre corchetes [  ]. Los argumentos no opcionales van simplemente después de una comma.
  En este caso tendremos nuestro comando de la siguiente forma:
  \begin{lstlisting}
    capture program drop generate_name
    program define generate_name
      syntax varname, Generate(name)
      ...
    end
  \end{lstlisting}

  Ahora nosotros tendremos dos macros que podemos utilizar en nuestro comando: "varlist" y "generate".

\end{frame}

%-----------------------------------------------
\begin{frame}[t, fragile]{Creando nuestros comandos: [Opciones]}
  Aqui es donde ya encontraremos las cosas más interesantes de los comandos. Por ejemplo:
  \begin{lstlisting}
    capture program drop generate_name
    program define generate_name
      syntax varname, generate(name)
        gen `generate' = `varlist'
        
    end
  \end{lstlisting}

  También podríamos agregar si queremos hacer cambios extras a esta nueva variable. Pasaremos a Stata ahora para realizar más ejemplos.
\end{frame}


%==============================================================
\miniframesoff 	

\begin{frame}[plain, noframenumbering]
	\begin{center}
	\LARGE STATA TIME
		\begin{figure}[H]
			\includegraphics[width=0.57\textwidth]{stata.pdf}
		\end{figure}
	\end{center}
\end{frame}
%==============================================================
% END
%==============================================================	
\begin{frame}[plain, standout]
	End.
\end{frame}
%-----------------------------------------------
\end{document}		
